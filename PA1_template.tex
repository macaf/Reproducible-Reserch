% Options for packages loaded elsewhere
\PassOptionsToPackage{unicode}{hyperref}
\PassOptionsToPackage{hyphens}{url}
%
\documentclass[
]{article}
\usepackage{lmodern}
\usepackage{amssymb,amsmath}
\usepackage{ifxetex,ifluatex}
\ifnum 0\ifxetex 1\fi\ifluatex 1\fi=0 % if pdftex
  \usepackage[T1]{fontenc}
  \usepackage[utf8]{inputenc}
  \usepackage{textcomp} % provide euro and other symbols
\else % if luatex or xetex
  \usepackage{unicode-math}
  \defaultfontfeatures{Scale=MatchLowercase}
  \defaultfontfeatures[\rmfamily]{Ligatures=TeX,Scale=1}
\fi
% Use upquote if available, for straight quotes in verbatim environments
\IfFileExists{upquote.sty}{\usepackage{upquote}}{}
\IfFileExists{microtype.sty}{% use microtype if available
  \usepackage[]{microtype}
  \UseMicrotypeSet[protrusion]{basicmath} % disable protrusion for tt fonts
}{}
\makeatletter
\@ifundefined{KOMAClassName}{% if non-KOMA class
  \IfFileExists{parskip.sty}{%
    \usepackage{parskip}
  }{% else
    \setlength{\parindent}{0pt}
    \setlength{\parskip}{6pt plus 2pt minus 1pt}}
}{% if KOMA class
  \KOMAoptions{parskip=half}}
\makeatother
\usepackage{xcolor}
\IfFileExists{xurl.sty}{\usepackage{xurl}}{} % add URL line breaks if available
\IfFileExists{bookmark.sty}{\usepackage{bookmark}}{\usepackage{hyperref}}
\hypersetup{
  pdftitle={Project 1},
  pdfauthor={Macarena Fernández},
  hidelinks,
  pdfcreator={LaTeX via pandoc}}
\urlstyle{same} % disable monospaced font for URLs
\usepackage[margin=1in]{geometry}
\usepackage{color}
\usepackage{fancyvrb}
\newcommand{\VerbBar}{|}
\newcommand{\VERB}{\Verb[commandchars=\\\{\}]}
\DefineVerbatimEnvironment{Highlighting}{Verbatim}{commandchars=\\\{\}}
% Add ',fontsize=\small' for more characters per line
\usepackage{framed}
\definecolor{shadecolor}{RGB}{248,248,248}
\newenvironment{Shaded}{\begin{snugshade}}{\end{snugshade}}
\newcommand{\AlertTok}[1]{\textcolor[rgb]{0.94,0.16,0.16}{#1}}
\newcommand{\AnnotationTok}[1]{\textcolor[rgb]{0.56,0.35,0.01}{\textbf{\textit{#1}}}}
\newcommand{\AttributeTok}[1]{\textcolor[rgb]{0.77,0.63,0.00}{#1}}
\newcommand{\BaseNTok}[1]{\textcolor[rgb]{0.00,0.00,0.81}{#1}}
\newcommand{\BuiltInTok}[1]{#1}
\newcommand{\CharTok}[1]{\textcolor[rgb]{0.31,0.60,0.02}{#1}}
\newcommand{\CommentTok}[1]{\textcolor[rgb]{0.56,0.35,0.01}{\textit{#1}}}
\newcommand{\CommentVarTok}[1]{\textcolor[rgb]{0.56,0.35,0.01}{\textbf{\textit{#1}}}}
\newcommand{\ConstantTok}[1]{\textcolor[rgb]{0.00,0.00,0.00}{#1}}
\newcommand{\ControlFlowTok}[1]{\textcolor[rgb]{0.13,0.29,0.53}{\textbf{#1}}}
\newcommand{\DataTypeTok}[1]{\textcolor[rgb]{0.13,0.29,0.53}{#1}}
\newcommand{\DecValTok}[1]{\textcolor[rgb]{0.00,0.00,0.81}{#1}}
\newcommand{\DocumentationTok}[1]{\textcolor[rgb]{0.56,0.35,0.01}{\textbf{\textit{#1}}}}
\newcommand{\ErrorTok}[1]{\textcolor[rgb]{0.64,0.00,0.00}{\textbf{#1}}}
\newcommand{\ExtensionTok}[1]{#1}
\newcommand{\FloatTok}[1]{\textcolor[rgb]{0.00,0.00,0.81}{#1}}
\newcommand{\FunctionTok}[1]{\textcolor[rgb]{0.00,0.00,0.00}{#1}}
\newcommand{\ImportTok}[1]{#1}
\newcommand{\InformationTok}[1]{\textcolor[rgb]{0.56,0.35,0.01}{\textbf{\textit{#1}}}}
\newcommand{\KeywordTok}[1]{\textcolor[rgb]{0.13,0.29,0.53}{\textbf{#1}}}
\newcommand{\NormalTok}[1]{#1}
\newcommand{\OperatorTok}[1]{\textcolor[rgb]{0.81,0.36,0.00}{\textbf{#1}}}
\newcommand{\OtherTok}[1]{\textcolor[rgb]{0.56,0.35,0.01}{#1}}
\newcommand{\PreprocessorTok}[1]{\textcolor[rgb]{0.56,0.35,0.01}{\textit{#1}}}
\newcommand{\RegionMarkerTok}[1]{#1}
\newcommand{\SpecialCharTok}[1]{\textcolor[rgb]{0.00,0.00,0.00}{#1}}
\newcommand{\SpecialStringTok}[1]{\textcolor[rgb]{0.31,0.60,0.02}{#1}}
\newcommand{\StringTok}[1]{\textcolor[rgb]{0.31,0.60,0.02}{#1}}
\newcommand{\VariableTok}[1]{\textcolor[rgb]{0.00,0.00,0.00}{#1}}
\newcommand{\VerbatimStringTok}[1]{\textcolor[rgb]{0.31,0.60,0.02}{#1}}
\newcommand{\WarningTok}[1]{\textcolor[rgb]{0.56,0.35,0.01}{\textbf{\textit{#1}}}}
\usepackage{graphicx,grffile}
\makeatletter
\def\maxwidth{\ifdim\Gin@nat@width>\linewidth\linewidth\else\Gin@nat@width\fi}
\def\maxheight{\ifdim\Gin@nat@height>\textheight\textheight\else\Gin@nat@height\fi}
\makeatother
% Scale images if necessary, so that they will not overflow the page
% margins by default, and it is still possible to overwrite the defaults
% using explicit options in \includegraphics[width, height, ...]{}
\setkeys{Gin}{width=\maxwidth,height=\maxheight,keepaspectratio}
% Set default figure placement to htbp
\makeatletter
\def\fps@figure{htbp}
\makeatother
\setlength{\emergencystretch}{3em} % prevent overfull lines
\providecommand{\tightlist}{%
  \setlength{\itemsep}{0pt}\setlength{\parskip}{0pt}}
\setcounter{secnumdepth}{-\maxdimen} % remove section numbering

\title{Project 1}
\author{Macarena Fernández}
\date{December 28, 2020}

\begin{document}
\maketitle

\#Project 1 by Macarena Fernández

Load the data:

\begin{Shaded}
\begin{Highlighting}[]
\KeywordTok{library}\NormalTok{(knitr)}
\end{Highlighting}
\end{Shaded}

\begin{verbatim}
## Warning: package 'knitr' was built under R version 3.6.3
\end{verbatim}

\begin{Shaded}
\begin{Highlighting}[]
\KeywordTok{library}\NormalTok{(readr)}
\end{Highlighting}
\end{Shaded}

\begin{verbatim}
## Warning: package 'readr' was built under R version 3.6.3
\end{verbatim}

\begin{Shaded}
\begin{Highlighting}[]
\NormalTok{activity <-}\StringTok{ }\KeywordTok{read_csv}\NormalTok{(}\StringTok{"C:/Users/pc/Desktop/Python/activity.csv"}\NormalTok{)}
\end{Highlighting}
\end{Shaded}

\begin{verbatim}
## 
## -- Column specification ------------------------------------------------------------------------
## cols(
##   steps = col_double(),
##   date = col_date(format = ""),
##   interval = col_double()
## )
\end{verbatim}

Histograms of steps per day:

\begin{Shaded}
\begin{Highlighting}[]
\NormalTok{s_by_d=}\KeywordTok{tapply}\NormalTok{(activity}\OperatorTok{$}\NormalTok{steps,}\KeywordTok{as.factor}\NormalTok{(activity}\OperatorTok{$}\NormalTok{date),sum,}\DataTypeTok{na.rm=}\OtherTok{TRUE}\NormalTok{)}
\KeywordTok{hist}\NormalTok{(s_by_d,}\DataTypeTok{main=}\StringTok{"Histograms of steps by day"}\NormalTok{,}\DataTypeTok{xlab=}\StringTok{"Steps by day"}\NormalTok{,}\DataTypeTok{col=}\StringTok{"green"}\NormalTok{)}
\end{Highlighting}
\end{Shaded}

\includegraphics{PA1_template_files/figure-latex/unnamed-chunk-2-1.pdf}

\begin{Shaded}
\begin{Highlighting}[]
\KeywordTok{summary}\NormalTok{(s_by_d)}
\end{Highlighting}
\end{Shaded}

\begin{verbatim}
##    Min. 1st Qu.  Median    Mean 3rd Qu.    Max. 
##       0    6778   10395    9354   12811   21194
\end{verbatim}

Time series plot of the average of steps taken by day:

\begin{Shaded}
\begin{Highlighting}[]
\NormalTok{m_by_i=}\KeywordTok{tapply}\NormalTok{(activity}\OperatorTok{$}\NormalTok{steps,}\KeywordTok{as.factor}\NormalTok{(activity}\OperatorTok{$}\NormalTok{interval),mean,}\DataTypeTok{na.rm=}\OtherTok{TRUE}\NormalTok{)}
\NormalTok{intervals=}\KeywordTok{unique}\NormalTok{(activity}\OperatorTok{$}\NormalTok{interval)}
\KeywordTok{plot}\NormalTok{(intervals,m_by_i,}\DataTypeTok{type=}\StringTok{"l"}\NormalTok{,}\DataTypeTok{main=}\StringTok{"Time series average steps"}\NormalTok{,}\DataTypeTok{xlab=}\StringTok{"Intervals"}\NormalTok{,}\DataTypeTok{ylab=}\StringTok{"Mean Steps"}\NormalTok{)}
\end{Highlighting}
\end{Shaded}

\includegraphics{PA1_template_files/figure-latex/unnamed-chunk-3-1.pdf}
The 5-minute interval that, on average, contains the maximum number of
steps:

\begin{Shaded}
\begin{Highlighting}[]
\NormalTok{i=}\KeywordTok{which.max}\NormalTok{(m_by_i)}
\KeywordTok{names}\NormalTok{(m_by_i)[i]}
\end{Highlighting}
\end{Shaded}

\begin{verbatim}
## [1] "835"
\end{verbatim}

Number of missings Values:

\begin{Shaded}
\begin{Highlighting}[]
\KeywordTok{sum}\NormalTok{(}\KeywordTok{is.na}\NormalTok{(activity}\OperatorTok{$}\NormalTok{steps))}
\end{Highlighting}
\end{Shaded}

\begin{verbatim}
## [1] 2304
\end{verbatim}

Straregy for imputing missing values: mean of the interval

\begin{Shaded}
\begin{Highlighting}[]
\ControlFlowTok{for}\NormalTok{ (i }\ControlFlowTok{in} \DecValTok{1}\OperatorTok{:}\KeywordTok{nrow}\NormalTok{(activity))\{}
  \ControlFlowTok{if}\NormalTok{ (}\KeywordTok{is.na}\NormalTok{(activity[i,}\DecValTok{1}\NormalTok{]))\{}
\NormalTok{    interval=}\KeywordTok{as.numeric}\NormalTok{(activity[i,}\DecValTok{3}\NormalTok{])}
\NormalTok{    ind=}\KeywordTok{which}\NormalTok{(intervals}\OperatorTok{==}\NormalTok{interval)}
\NormalTok{    activity[i,}\DecValTok{1}\NormalTok{]=m_by_i[[ind]]}
    
\NormalTok{  \}}
\NormalTok{\}}
\end{Highlighting}
\end{Shaded}

Histogram:

\begin{Shaded}
\begin{Highlighting}[]
\NormalTok{s_by_d=}\KeywordTok{tapply}\NormalTok{(activity}\OperatorTok{$}\NormalTok{steps,}\KeywordTok{as.factor}\NormalTok{(activity}\OperatorTok{$}\NormalTok{date),sum)}
\KeywordTok{hist}\NormalTok{(s_by_d,}\DataTypeTok{main=}\StringTok{"Histograms of steps by day"}\NormalTok{,}\DataTypeTok{xlab=}\StringTok{"Steps by day"}\NormalTok{,}\DataTypeTok{col=}\StringTok{"blue"}\NormalTok{)}
\end{Highlighting}
\end{Shaded}

\includegraphics{PA1_template_files/figure-latex/unnamed-chunk-7-1.pdf}

\begin{Shaded}
\begin{Highlighting}[]
\KeywordTok{summary}\NormalTok{(s_by_d)}
\end{Highlighting}
\end{Shaded}

\begin{verbatim}
##    Min. 1st Qu.  Median    Mean 3rd Qu.    Max. 
##      41    9819   10766   10766   12811   21194
\end{verbatim}

Factor Variable 0 is the date is a weekday 1 if is a weekend:

\begin{Shaded}
\begin{Highlighting}[]
\NormalTok{activity}\OperatorTok{$}\NormalTok{weekdays=}\DecValTok{0}
\NormalTok{weekend=}\KeywordTok{c}\NormalTok{(}\StringTok{"sábado"}\NormalTok{, }\StringTok{"domingo"}\NormalTok{)}
\NormalTok{activity}\OperatorTok{$}\NormalTok{weekdays[}\KeywordTok{weekdays}\NormalTok{(activity}\OperatorTok{$}\NormalTok{date) }\OperatorTok\StringTok{ }\NormalTok{weekend]=}\DecValTok{1}
\NormalTok{activity}\OperatorTok{$}\NormalTok{weekdays=}\KeywordTok{factor}\NormalTok{(activity}\OperatorTok{$}\NormalTok{weekdays,}\DataTypeTok{labels=}\KeywordTok{c}\NormalTok{(}\StringTok{"weekday"}\NormalTok{,}\StringTok{"weekend"}\NormalTok{))}
\end{Highlighting}
\end{Shaded}

Plots of average steps by interval, one for weekdays and weekend:

\begin{Shaded}
\begin{Highlighting}[]
\NormalTok{weekend=}\KeywordTok{subset}\NormalTok{(activity,weekdays}\OperatorTok{==}\StringTok{"weekend"}\NormalTok{)}
\NormalTok{weekday=}\KeywordTok{subset}\NormalTok{(activity,weekdays}\OperatorTok{==}\StringTok{"weekday"}\NormalTok{)}
\NormalTok{m_by_i_we=}\KeywordTok{tapply}\NormalTok{(weekend}\OperatorTok{$}\NormalTok{steps,}\KeywordTok{as.factor}\NormalTok{(weekend}\OperatorTok{$}\NormalTok{interval),mean)}
\NormalTok{m_by_i_wd=}\KeywordTok{tapply}\NormalTok{(weekday}\OperatorTok{$}\NormalTok{steps,}\KeywordTok{as.factor}\NormalTok{(weekday}\OperatorTok{$}\NormalTok{interval),mean)}
\NormalTok{i_we=}\KeywordTok{unique}\NormalTok{(weekend}\OperatorTok{$}\NormalTok{interval)}
\NormalTok{i_wd=}\KeywordTok{unique}\NormalTok{(weekday}\OperatorTok{$}\NormalTok{interval)}
\NormalTok{r=}\KeywordTok{range}\NormalTok{(m_by_i_wd,m_by_i_we)}
\end{Highlighting}
\end{Shaded}

Graph:

\begin{Shaded}
\begin{Highlighting}[]
\KeywordTok{par}\NormalTok{(}\DataTypeTok{mfrow=}\KeywordTok{c}\NormalTok{(}\DecValTok{2}\NormalTok{,}\DecValTok{1}\NormalTok{),}\DataTypeTok{mar =} \KeywordTok{c}\NormalTok{(}\DecValTok{4}\NormalTok{,}\DecValTok{2}\NormalTok{,}\DecValTok{4}\NormalTok{,}\DecValTok{2}\NormalTok{))}
\KeywordTok{plot}\NormalTok{(i_we,m_by_i_we,}\DataTypeTok{type=}\StringTok{"l"}\NormalTok{,}\DataTypeTok{col=}\StringTok{"blue"}\NormalTok{,}\DataTypeTok{main=}\StringTok{"Weekend"}\NormalTok{,}\DataTypeTok{xlab=}\StringTok{"interval"}\NormalTok{,}\DataTypeTok{ylab=}\StringTok{"step"}\NormalTok{,}\DataTypeTok{ylim=}\NormalTok{r)}
\KeywordTok{plot}\NormalTok{(i_wd,m_by_i_wd,}\DataTypeTok{type=}\StringTok{"l"}\NormalTok{,}\DataTypeTok{col=}\StringTok{"blue"}\NormalTok{,}\DataTypeTok{main=}\StringTok{"Weekend"}\NormalTok{,}\DataTypeTok{xlab=}\StringTok{"interval"}\NormalTok{,}\DataTypeTok{ylab=}\StringTok{"step"}\NormalTok{,}\DataTypeTok{ylim=}\NormalTok{r)}
\end{Highlighting}
\end{Shaded}

\includegraphics{PA1_template_files/figure-latex/unnamed-chunk-10-1.pdf}

\end{document}
